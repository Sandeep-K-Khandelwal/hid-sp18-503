% status: 20%
% chapter: TBD

\title{Apache Milagro}


\author{Arnav Arnav}
\affiliation{%
  \institution{Indiana University
  Bloomington} \city{Bloomington} \state{Indiana} \postcode{47408} \country{USA}
  }
\email{aarnav@iu.edu}

% The default list of authors is too long for headers}
\renewcommand{\shortauthors}{Arnav}


\begin{abstract}
An increasing number of devices and applications, rely on cloud-based
app-centric services. Most of these application, such as Internet of
Things (IoT) applications rely on traditional Public Key
Infrastructure (PKI), to securely communicate the data. It is known
that PKI infrastucture has problems and is not the best solution as
the number of connected devices increses.  Apache Miilagro provides a
distributed and scalable multi factor authentication that is ideal for
app-centric services.
\end{abstract}

\keywords{hid-sp18-503, Milagro, cryptography, cloud applications, IoT }
%use up to 5 keywords

\maketitle

%% kaafi images
\section{Introduction}

%shift towards app centric services
%more connected devices
% why MFA is needed
%apache milagro solution
%Miracl, MFA, TLS intro


\section{Problems with PKI}
Public Key Infrastructure has had many problems and has faced various
threats over the years.  Firstly, PKI has too many moving parts, and
setting up a PKI service is not a trivial task. This has lead to many
organizaations not deploying PKI correctly and leaving their systems
vulnerable. Furthermore, most applications and users ignore PKI
warnings such as a website not being secure. These warnings and errors
can not be enforced in web browsers now, otherwise a lot of pages and
services would stop working~\cite{cso-pki-problems}.

Since distibuting the list of revoked certificates to the whole system
can take a long time, there is no guarantee that a key belongs to a
well identified user at any time. Users may have certificates from
different Certificate Authorities, which makes it difficult to
identify users that have revoked certificates form some Certificate
Authorities~\cite{distlab-pki-problems}.  There have been attacks on
PKI Certificate Authorities (CAs) and various CAs have been
compromised. Not only this, atackers in the past have been successful
in stealing website sertificates, and code signing certificates as in
the case of stuxnet~\cite{securityweek-ssl-threats}. Encryption makes
initiating an SSL connection a heavyweight process and thus SSl
prottected services are susceptible to attacks such as Denial of
Servics (DoS) and Distributed Denial of Service (DDoS). Many such
attacks on PKI hvae been listed in ~\cite{cacert-wiki-pki-history}.

Most password protected services rely on PKI and SSL to send passwords
over the internet. Recording passwords on a server is another problem
as password servers can be attacked and passwords can be stolen even
if PKI and SSL are working propoerly. A morephilosophical issue with
centralized PKI is that the user identificationinformation lies with
centralized servers and not users
themselves~\cite{distlab-pki-poblems}.  Various approaches have been
proposed to avoid some of these errors, none of which have been widely
accepted. Google has proposed enhancement to PKI as they suspect that
PKI may be broken in the future and attackers can use this to decipher
recorded messages from old
communications~\cite{securityweek-ssl-threats}.  Many of these
problems can be avoided by adopting a decentralized approach.


\section{Milagro Details}
% why milagro
% how it works
% shared keys, d-ta and stuff
% extend trust as needed
% revoke trust

\section{Miracl}
% details on miracl
% how it is used in milagro

\section{Milagro TLS}
% perfect forward secrecy
% certificateless TLS
% using Milagro auth for peer to peer security
% how it works

\section{case study} % future work
% uk tax office user identity assurance
% etc
\section{conclusion}

\begin{acks}

  The authors would like to thank Dr.~Gregor~von~Laszewski for his
  support and suggestions to write this paper.

\end{acks}


%\begin{figure}[!ht]
%  \centering\includegraphics[width=\columnwidth]{../../hid-sample/tex/images/fly.pdf}
%  \caption{Example caption}\label{f:fly}
%\end{figure}



\bibliographystyle{ACM-Reference-Format}
\bibliography{report} 
