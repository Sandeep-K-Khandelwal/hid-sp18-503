% status: 20%
% chapter: TBD

\title{Apache Milagro}


\author{Arnav Arnav}
\affiliation{%
  \institution{Indiana University
  Bloomington} \city{Bloomington} \state{Indiana} \postcode{47408} \country{USA}
  }
\email{aarnav@iu.edu}

% The default list of authors is too long for headers}
\renewcommand{\shortauthors}{Arnav}


\begin{abstract}
An increasing number of devices and applications, rely on cloud-based
app-centric services. Most of these application, such as Internet of
Things (IoT) applications rely on traditional Public Key
Infrastructure (PKI), to securely communicate the data. It is known
that PKI infrastucture has problems and is not the best solution as
the number of connected devices increses.  Apache Miilagro provides a
distributed and scalable multi factor authentication that is ideal for
app-centric services.
\end{abstract}

\keywords{hid-sp18-503, Milagro, cryptography, cloud applications, IoT }
%use up to 5 keywords

\maketitle

%% kaafi images
\section{Introduction}
With the increasing competition in the industry, fast delivery of
services to users is one of the major concerns. Starting up services
with traditional cloud Infrastructure as a Service (IaaS) platforms
can be challenging since bulk of the work needed to set up and deploy
an application is left for the user. With the increase in application
centric cloud computing, the developers do not need to worry about all
of the networking issues~\cite{bmc-app-centric}. An application
centric cloud computing environment abstracts away the complexities of
different cloud environments, allowing the developers need to manage
specific tasks related to the application and not the individual
servers themselves, thus enabling developers to spend more time to
improve the overall user experience. Most IoT applications benefit
from application centric clouds as the connected devices are part of a
common application~\cite{vb-app-centric}.

In view of this, it is important to ensure security in the cloud. With
most of the services using password based authentication, which is
inherently vulnerable, it is important to use other information to
authenticate users properly. Multi Factor Authentication (MFA) allows
users to establish their identity by providing evidence for two out of
three requirements- knowledge (something the user knows), posession
(something the user has) and inherence (something the user
is)~\cite{centrify-mfa}.

Apache Milagro~\cite{milagro-website} is an apache incubator project
that provides an open source, distributd, certificate less security
solution for cloud based application that is easy to scale. Milagro
consists of Apache Milagro Crypto Library (AMCL) that allows users to
establish distributed trust systems using proven and tested
propocols. Applications can then build on MCL and provide multifactor
authentication using Milagro MFA, secure communications between a
client and a server and also in peer to peer
comunication~\cite{milagro-docs-overview}.



\section{Problems with PKI}
Public Key Infrastructure has had many problems and has faced various
threats over the years.  Firstly, PKI has too many moving parts, and
setting up a PKI service is not a trivial task. This has lead to many
organizaations not deploying PKI correctly and leaving their systems
vulnerable. Furthermore, most applications and users ignore PKI
warnings such as a website not being secure. These warnings and errors
can not be enforced in web browsers now, otherwise a lot of pages and
services would stop working~\cite{cso-pki-problems}.

Since distibuting the list of revoked certificates to the whole system
can take a long time, there is no guarantee that a key belongs to a
well identified user at any time. Users may have certificates from
different Certificate Authorities, which makes it difficult to
identify users that have revoked certificates form some Certificate
Authorities~\cite{distlab-pki-problems}.  There have been attacks on
PKI Certificate Authorities (CAs) and various CAs have been
compromised. Not only this, atackers in the past have been successful
in stealing website sertificates, and code signing certificates as in
the case of stuxnet~\cite{securityweek-ssl-threats}. Encryption makes
initiating an SSL connection a heavyweight process and thus SSl
prottected services are susceptible to attacks such as Denial of
Servics (DoS) and Distributed Denial of Service (DDoS). Many such
attacks on PKI hvae been listed in~\cite{cacert-wiki-pki-history}.
Many commercial PKI services use Identity certificates to sidestep the
concern whether a commercial CA actualy had the authority to provide
certificates, and leave it to the users to verify the actual
identity~\cite{ten-pki-risks}.

Most password protected services rely on PKI and SSL to send passwords
over the internet. Recording passwords on a server is another problem
as password servers can be attacked and passwords can be stolen even
if PKI and SSL are working propoerly. A morephilosophical issue with
centralized PKI is that the user identificationinformation lies with
centralized servers and not users
themselves~\cite{distlab-pki-problems}.  Various approaches have been
proposed to avoid some of these errors, none of which have been widely
accepted. Google has proposed enhancement to PKI as they suspect that
PKI may be broken in the future and attackers can use this to decipher
recorded messages from old
communications~\cite{securityweek-ssl-threats}.  Many of these
problems can be avoided by adopting a decentralized approach.


\section{Milagro Details}
% more on elliptic curve pairing -- medium
Milagro uses cryptography techniques based on pairing on ellipic
curves. Elliptic curves are mathematical structures on which
operations are easily defined. Muliplication of a point on a curve to
a number can be easily computed, but it is computationally difficult
to find the multiplier even if the result is known. This makes it
difficult to break the system while using smaller multipliers as
compared to RSA based encryption which requires generation of large
prime numbers~\cite{milagro-concepts}. This removes the need to
maintain and distribute certificates, thus eliminating the need for a
centralized certificate authority.

Apache milagro uses pairing based cryptography (PBC) to split a users
key among different distributed trust authorities (D-TAs) and provides
additional security as the D-TAs are isolated and idependent of each
other. This eliminates a single point of failure, and allows the key
generation and distribution to be a part of the system itself. Apache
milagro provides Identity Based Encryption (IBE) where a users key can
be set to a predefined identifier, and ero knowledge authentication,
where the identity of a user can be proved without showing the user's
key.  These cryptographic operations are handled by the
MCL~\cite{milagro-concepts}.

\section{Apache Milagro Crypto Library}
% important features of mcl
Despite extensive research in cryptography, most of the industry
relies on old technology for security solutions. There are many
libreries available that provide new cryptographic methods that
attempt to solve the issues in PKI. Most of these libraries are not
easy to use, or are dependent on other
liraries~\cite{mcl-white-paper}.

Apache Milagro crypto library is
self contained and pnly dependes on external processes to add
randomness to the key generation process. The library is portable as
it is not written in assembly language and is reasonably fast. The
library has been created keeping in mind the memory restrictions on
small connected devices and takes minimum space. AMCL is available in
many languages namely C, Java, Javascript, Swift and Go, and uses
general programming constructs avoiding language specific
functions. This allows the library to be transated to most other
langages using already available translators~\cite{mcl-white-paper}.

AMCL uses 128-bit AES encryption, since it is the current standard for
cryptography and works as well as other variants like 256-bit AES, or
192-bit AES. Apart from this, the library uses ``SHA256 for hashing,
256-bit prime field elliptic curves for public key protocols, and
256-bit BN curves to support pairing-based protocols. However three
different parameterizations of Elliptic curve are supported -
Weierstrass, Edwards and Montgomery, as each is appropriate within its
own niche''~\cite{mcl-white-paper}. The library borrows random number
generation and symmetric encryption from the open source MIRACL
library\cite{mcl-white-paper}.


\section{Milagro MFA}
% how it works
% how it builds on mcl
% different sdks available

\section{Milagro TLS}
% perfect forward secrecy
% certificateless TLS
% using Milagro auth for peer to peer security
% how it works

%\section{case study} % future work
% uk tax office user identity assurance
% etc
\section{conclusion}

%\begin{figure}[!ht]
%  \centering\includegraphics[width=\columnwidth]{../../hid-sample/tex/images/fly.pdf}
%  \caption{Example caption}\label{f:fly}
%\end{figure}

\begin{acks}

  The authors would like to thank Dr.~Gregor~von~Laszewski for his
  support and suggestions to write this paper.

\end{acks}



\bibliographystyle{ACM-Reference-Format}
\bibliography{report} 
