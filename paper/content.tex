% status: 20%
% chapter: TBD

\title{Big Data for Edge Computing}


\author{Arnav Arnav}
\affiliation{%
  \institution{Indiana University Bloomington}
  \city{Bloomington} 
  \state{Indiana} 
  \postcode{47408}
  \country{USA}
}
\email{aarnav@iu.edu}

% The default list of authors is too long for headers}
\renewcommand{\shortauthors}{Arnav}


\begin{abstract}
An increasing number of devices and applications, rely on cloud-based
app-centric services. Most of these application, such as Internet of
Things (IoT) applications rely on traditional Public Key
Infrastructure (PKI), to securely communicate the data. It is known
that PKI infrastucture has problems and is not the best solution as
the number of connected devices increses.  Apache Miilagro provides a
distributed and scalable multi factor authentication that is ideal for
app-centric services.
\end{abstract}

\keywords{hid-sp18-503, Milagro, cryptography, cloud applications, IoT }%use up to 5 keywords

\maketitle

\section{Introduction}

%shift towards app centric services
%more connected devices
%problems with PKI
%apache milagro solution
%Miracl, MFA, TLS intro

\section{Problems with PKI}
% root key compromize
% session issues
% historical examples

\section{Milagro Details}
% why milagro
% how it works
% shared keys, d-ta and stuff
% extend trust as needed
% revoke trust

\section{Miracl}
% details on miracl
% how it is used in milagro

\section{Milagro TLS}
% perfect forward secrecy
% certificateless TLS
% using Milagro auth for peer to peer security
% how it works

\section{case study} % future work
% uk tax office user identity assurance
% etc
\section{conclusion}

\begin{acks}

  The authors would like to thank Dr.~Gregor~von~Laszewski for his
  support and suggestions to write this paper.

\end{acks}


\section{Figures}

%\begin{figure}[!ht]
%  \centering\includegraphics[width=\columnwidth]{../../hid-sample/tex/images/fly.pdf}
%  \caption{Example caption}\label{f:fly}
%\end{figure}


\section{Check}

make sure just as in previous assignments that you check your paper
with chktex and lacheck. Fix the errors that you see. Some of the
errors may be ok, but in general make sure you address all of them. If
in doubt work with the TA. Simply use

\begin{verbatim}
make check
\end{verbatim}

\begin{verbatim}
make check-ta
\end{verbatim}

\bibliographystyle{ACM-Reference-Format}
\bibliography{report} 
