\section{Apache Carbondata}
\index{Carbondata}
\index{Apache Carbondata}
%\index{Monitoring!XDMoD}
%\index{Monitoring!TAC\_Stat}


As the amount of data we have increases storing and performing
analytics of this data becomes increasingly difficult.
Apache carbondata is an indexed file format for storing big data
that allows faster analysis on a huge amounts of data
\cite{hid-sp18-503-carbondata-docs}. Carbondata runs on top of hadoop YARN
and spark and can be uses columnar storage, compression and
encoding techniques to perform faster queries on the data.

An Apache Carbondata file system consists of groups of data called blocklets
and stores information like schema, in the header and footer co-located in HDFS.
The Footer is read once to create the index which is later utilized to
optimize queries \cite{hid-sp18-503-carbondata-docs}.

Apache Carbondata allows operations like creating tables, updating and
deleting them and performing queries on these tables
\cite{hid-sp18-503-carbondata-mgmt}.



